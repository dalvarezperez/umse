\begin{abstract}
  Malware sample confidentiality is an unsolved problem which affects the
  major security products including antiviruses, threat hunting and threat
  intelligence products\cite{EuropeanParliamentReportOnCyberDefence}\cite{EuropeanParliamentDesignatingAsDangerous} which, always, in greater or lesser volume,
  potentially collect confidential elements\cite{Kaspersky2020EulaEn}\cite{McAfeeEnterpriseEulaEnGb}\cite{MalwarebytesEula} (including files, system events,
  network traffic because of firewall capabilities, etc.). These are not
  necessary malware, but are also paradoxically called malware samples. This
  happens because new knowledge cannot come but from other place than unknown
  and undetected elements. We consider this a critical issue because the
  capabilities of surveillance of free and paid security products (including
  those that come built-in the operating system) are huge\cite{KasperskyBoundariesOfTrust} and growing\cite{ThreatHuntingForDummies}\cite{ProactiveHunting} in terms
  of local system and local area network access.  The solution proposed in
  this work to this confidentiality risk and access logging problem involves
  the design of a new encrypted malware sample format: Universal Malware
  Sample Encryption (UMSE).  UMSE is a rich format. It can represent: software
  threats, hardware threats\cite{HardwareMalware}, the mixing of both and takes into account very
  important forgotten aspects like: potential malicious elements context\cite{ContextBasedMalware}\cite{LearningFromContext}, life
  cycle\cite{SoftwareEngineering} and variety of nature of the elements (not limited to files\cite{ResidentViruses}\cite{FilelessAttacks}\cite{AdvacedVolatileThreat}) and some
  other things which finally improve the sample quality acquisition, storage
  and transport positively impacting all subsequent reverse engineering tasks
  and users confidentiality.
\end{abstract}
