% Some commands used in this file
\newcommand{\package}{\emph}

\chapter{Introduction}

Understanding by malware sample the evidence set of any kind of malicious
element, hardware and/or software, in a vital cycle state and belonging a
context, the present work addresses the malware sample process: acquisition,
efficient storage, confidentiality and access logging problem.

The result of this work is an encrypted malware sample format called Universal
Malware Sample Encryption (hereafter, UMSE). The development, far from being
limited to the sheer format specification, consisted in the following:
\begin{itemize}
\item UMSE format documentation and specification.
\item A Microsoft Windows Portable Executable\cite{MicrosoftPecoff} dynamic linking library\cite{DynamicLinkLibraries}
  implementing all necessary functions to work with UMSE, for instance:
  generate an UMSE malware sample, decrypt some sample parts regarding
  individual parts confidentiality, etc.  This library was developed in C/C++
  to be used by security products which are mostly developed in these
  languages (well, although it can be called from other languages).
\item A very elementary antimalware agent simulator which acquires system
  elements demonstrating how easy is to integrate the UMSE dynamic linking
  library with existing security products.
\item An intelligence tool. It consists in a web panel allowing to operate and
  manage UMSE malware samples stored in a generic database. All malware
  samples come from the mentioned antimalware agent simulator, which
  recollects potential malicious elements and sends them to this server. On
  the other hand, by virtue of UMSE format, each operation over the samples
  can be logged, as is this case here.
\item A Shell, allowing the malware analyst to communicate with the
  intelligence tool to work with samples. Possible operations are some of
  which, at least two, deserve mention: UMSE sample downloading and UMSE
  sample parts decryption in case that the analyst was sufficiently privileged
  in comparison to, not only the global, but also the confidentiality level
  for the specific individual sample parts.
\item A tool to quickly and easily generate samples, useful when encryption is
  not a must.
\end{itemize}
Below is the status of current malware sample acquisition, storage and
confidentiality in order to show the problem addressed in the rest of this
work.

% Structure of the document.