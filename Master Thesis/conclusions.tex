\chapter{Conclusions and future work}

The UMSE malware sample format improves malware reverse engineering processes\cite{ContextBasedMalware}\cite{LearningFromContext}\cite{ResidentViruses}\cite{FilelessAttacks}\cite{AdvacedVolatileThreat}
and also allows security products (including those which comes built-in the
operating system) to be polite with potentially confidential elements and
information about its clients and users.  Antivirus enterprises do not seem to
be interested in spying the user\cite{BBCMarcinKleczynski}, but maybe they collaborate with a third
party who also uses intelligence information for purposes unrelated to malware
analysis\cite{KasperskyBoundariesOfTrust}. By using UMSE, antivirus products can take the malware sample
treatment control in a transparent way and also protect users' data from a
hypothetical malicious third party.  With the arrival of cyber threat hunting
products (extremely aggressive products against confidentiality), antivirus
comparison tools must take seriously into account the issues of
confidentiality. The rest of features are already sufficiently competed.
Therefore, clients and users must take confidentiality into account while
choosing between them.  It is possible to keep the protection ratio and users
confidentiality at the same time if UMSE is used to acquire, transport and
store the samples.

There of course some developments and features which were not implemented in
UMSE so far. among the possible extensions that the tool could integrate, we
would like to mention the following ideas for future work:
\begin{itemize}
\item UMSE version 1.0 does not specify malware samples metadata. In order to
  be as flexible as possible, it lets this definition absolutely open to
  security tools which implements UMSE. The future work consist in to develop
  a sufficiently general malware UMSE metadata standard to reduce security
  tools UMSE implementation curve.
\item Current UMSE dynamic linking library includes some target format to UMSE
  converters \texttt{xxxToUmse.cpp}. For instance, \texttt{peToUMSE.cpp} was
  developed. We recognize that the mentioned converter file is too simple
  (special cases could lead to errors) and does not take advantage of Portable
  Executable granularity (for instance, it does not encrypt each binary
  resource separately). We let this improvement and the incorporation of new
  converters for a future work.
\item The main purpose of this work is to uncover some shortcomings in the
  current techniques for treating malware samples, and to provide a solution
  to them. Unfortunately, for reasons of time, we could not incorporate this
  mechanism to any existing open source solution for antimalware, like
  Clamav\footnote{\href{https://github.com/Cisco-Talos/clamav-devel}{\texttt{https://github.com/Cisco-Talos/clamav-devel}}}
  for instance, letting it also open for a future work.
\item Improve and add support of metadata and heterogeneous element per sample
  to ``Simple UMSE tool for single users'' mentioned in section XXX.
\end{itemize}